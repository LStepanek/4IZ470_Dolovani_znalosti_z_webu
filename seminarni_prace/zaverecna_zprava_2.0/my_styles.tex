%%%%%%%%%%%%%%%%%%%%%%%%%%%%%%%%%%%%%%%%%%%%%%%%%%%%%%%%%%%%%%%%%%%%%%%%%%%%%%%
%%%%%%%%%%%%%%%%%%%%%%%%%%%%%%%%%%%%%%%%%%%%%%%%%%%%%%%%%%%%%%%%%%%%%%%%%%%%%%%
%%%%%%%%%%%%%%%%%%%%%%%%%%%%%%%%%%%%%%%%%%%%%%%%%%%%%%%%%%%%%%%%%%%%%%%%%%%%%%%

% loaduji některé balíčky -----------------------------------------------------
\usepackage[T1]{fontenc}
\usepackage[utf8]{inputenc}
\usepackage[english, czech]{babel}

\usepackage[
  inner = 4cm,
  outer = 2cm,
  top = 3cm,
  bottom = 3cm
]{geometry}

\usepackage{amsmath}
\usepackage{mathtools}
\usepackage{amsfonts}
\usepackage{amssymb}
\usepackage{amsthm}

\usepackage{graphicx}

\usepackage{tocloft}
\usepackage{enumerate}
\usepackage{eso-pic}
\usepackage{csquotes}
\usepackage{float}
\usepackage[bottom]{footmisc}
\usepackage{caption}
\usepackage{array}
\usepackage{blindtext}
\usepackage{scrextend}
\usepackage{movie15}
\usepackage{caption}
\usepackage{epstopdf}
\usepackage{indentfirst}
\usepackage{imakeidx}
\usepackage{hyperref}
\usepackage{listings}
\usepackage{tikz}


% upřesňuji vlastnosti tikz obrázků -------------------------------------------
\usetikzlibrary{arrows,positioning} 
\tikzset{
    %Define standard arrow tip
    >=stealth',
    %Define style for boxes
    punkt/.style={
           rectangle,
           rounded corners,
           draw=black, very thick,
           text width=6.5em,
           minimum height=2em,
           text centered},
    % Define arrow style
    pil/.style={
           ->,
           thick,
           shorten <=2pt,
           shorten >=2pt,}
}
%\usepackage{fancyhdr}
\usepackage{titleps}


% nastavuji header ------------------------------------------------------------
\newpagestyle{my_page}{%
  \headrule
  \sethead[\sectiontitle][][]%
          {}{}{\sectiontitle}
  \setfoot[\thepage][][]%
          {}{}{\thepage}
}
\settitlemarks{section,subsection}
\pagestyle{my_page}

% nastavuji indentaci odstavců ------------------------------------------------
\setlength\parindent{24pt}

% nastavuji složku s grafikou -------------------------------------------------
\graphicspath{{./figures/}}

% bibliography management -----------------------------------------------------
\usepackage[
  backend = biber,
  style = numeric,
  bibencoding = UTF8,
  citestyle = ieee
]{biblatex}

\addbibresource{references.bib}    %% loaduji reference
\nocite{*}

% nastavuji parametry rejstříku -----------------------------------------------
\indexsetup{
  level = \section,
  toclevel = \section,
  noclearpage,
  firstpagestyle = my_page
}
\makeindex    %% inicializuji indexování do rejstříku

% nastavuji highlighting chunků kódu na jazyk R a další detaily ---------------
\captionsetup[lstlisting]{position = bottom}
\definecolor{my_gray}{rgb}{0.4,0.4,0.4}
\renewcommand{\lstlistingname}{Kód}  %% předefinovávám název chunků
\lstdefinestyle{custom_R}{
  basicstyle = \ttfamily\footnotesize,
  belowcaptionskip = 1\baselineskip,
  breaklines = true,
  frame = L,
  xleftmargin = \parindent,
  language = R,
  showstringspaces = false,
  keywordstyle = \bfseries\color{green!40!black},
  commentstyle = \itshape\color{purple!40!black},
  identifierstyle = \color{blue},
  stringstyle = \color{orange},
  numbers = left,
  numbersep = 12pt,
  numberstyle = \small\color{my_gray},
  texcl = true,
  alsoother = {\#},
  inputencoding = utf8,
  extendedchars = true,
  literate = %
  {á}{{\'a}}1
  {č}{{\v{c}}}1
  {ď}{{\v{d}}}1
  {é}{{\'e}}1
  {ě}{{\v{e}}}1
  {í}{{\'i}}1
  {ň}{{\v{n}}}1
  {ó}{{\'o}}1
  {ř}{{\v{r}}}1
  {š}{{\v{s}}}1
  {ť}{{\v{t}}}1
  {ú}{{\'u}}1
  {ů}{{\r{u}}}1
  {ý}{{\'y}}1
  {ž}{{\v{z}}}1
  {Á}{{\'A}}1
  {Č}{{\v{C}}}1
  {Ď}{{\v{D}}}1
  {É}{{\'E}}1
  {Ě}{{\v{E}}}1
  {Í}{{\'I}}1
  {Ň}{{\v{N}}}1
  {Ó}{{\'O}}1
  {Ř}{{\v{R}}}1
  {Š}{{\v{S}}}1
  {Ť}{{\v{T}}}1
  {Ú}{{\'U}}1
  {Ů}{{\r{U}}}1
  {Ý}{{\'Y}}1
  {Ž}{{\v{Z}}}1
}
\lstset{escapechar = @, style = custom_R}

% české uvozovky --------------------------------------------------------------
\DeclareQuoteAlias{german}{czech}
\MakeOuterQuote{"}

% přejmenovávám popisek obsahu na "Obsah" -------------------------------------
\renewcommand{\contentsname}{Obsah}

% přejmenovávám popisky obrázků na "Obr." -------------------------------------
\renewcommand{\figurename}{Obr.}

% přejmenovávám popisky tabulek na "Tab." -------------------------------------
\renewcommand\tablename{Tab.}

% upravuji formát obsahu ------------------------------------------------------
\renewcommand{\cftsecleader}{\cftdotfill{\cftdotsep}}

% definuji prostředí nečíslovaného lemmatu ------------------------------------
\newtheorem*{lemma*}{Lemma}

% handling pro správné číslování sekcí a subsekcí ve třídě "report" -----------
\renewcommand\thesection{\arabic{section}}

% číslování sekcí a subsekcí --------------------------------------------------
\setcounter{secnumdepth}{5}

% definuji arg max ------------------------------------------------------------
\DeclareMathOperator*{\argmax}{arg\,max}


%%%%%%%%%%%%%%%%%%%%%%%%%%%%%%%%%%%%%%%%%%%%%%%%%%%%%%%%%%%%%%%%%%%%%%%%%%%%%%%
%%%%%%%%%%%%%%%%%%%%%%%%%%%%%%%%%%%%%%%%%%%%%%%%%%%%%%%%%%%%%%%%%%%%%%%%%%%%%%%
%%%%%%%%%%%%%%%%%%%%%%%%%%%%%%%%%%%%%%%%%%%%%%%%%%%%%%%%%%%%%%%%%%%%%%%%%%%%%%%
