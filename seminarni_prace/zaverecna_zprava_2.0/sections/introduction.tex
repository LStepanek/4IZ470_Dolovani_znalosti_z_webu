%%%%%%%%%%%%%%%%%%%%%%%%%%%%%%%%%%%%%%%%%%%%%%%%%%%%%%%%%%%%%%%%%%%%%%%%%%%%%%%
%%%%%%%%%%%%%%%%%%%%%%%%%%%%%%%%%%%%%%%%%%%%%%%%%%%%%%%%%%%%%%%%%%%%%%%%%%%%%%%
%%%%%%%%%%%%%%%%%%%%%%%%%%%%%%%%%%%%%%%%%%%%%%%%%%%%%%%%%%%%%%%%%%%%%%%%%%%%%%%

\section{Úvod}

Účelem této minimalistické práce je nabídnout a popsat jeden z možných
způsobů, jak zpracovat relativně rozsáhlý, anglicky psaný text pro účely
jeho následné statistické reprezentace.

Autor ve výkladu systematicky popisuje proces zpracování textu přirozeného
jazyka od začátku do konce a bez zbytečného formalismu; v podstatě využívá
řešení úlohy nad konkrétními daty (korpusem), což posiluje praktický
aspekt celé práce.

Výchozím bodem pro zpracování přirozeného (anglického) textu je možnost
nakládat s textovým korpusem daného jazyka. Získaný korpus je poté v několika
na sebe navazujcích fázích postupně zpracováván tak, že z původních vět
korpusu nakonec vzniknou uspořádané seznamy slov a kratších sousloví
pevných délek, tzv. $n$-gramů\index{$n$-gram}. Pomocí nich pak lze jazyk
tzv. statisticky
reprezentovat, což umožňuje řešení několika zajímavých úloh nad přirozeným
jazykem. Jedna z nich je rovněž popsána -- jde o predikci $n$-tého slova%
\index{$n$-té slovo},
které by v daném jazyce mělo s největší pravděpodobností následovat po
zadané $(n - 1)$-členné frázi\index{$(n - 1)$-členná fráze}. Pro účely
real-time řešení této úlohy byla
publikována webová aplikace, jejíž popis a zdrojový kód je rovněž součástí
této publikace.

Funkce a procedury, jež řeší dále uvedené fáze zpracování textu korpusu, 
jsou implementovány v prostředí a programovacím jazyce \textsf{R}.
Znalost jazyka \textsf{R}\index{\textsf{R}} však není nutnou podmínkou pro
čtení této práce;
na následujících stránkách jsou popsány vesměs obecné principy zpracování
textu přirozeného jazyka a jsou uvedeny i obecné algoritmy.

V této práci je současně popsán i způsob, jak získat vlastní textový korpus,
opět s využitím již naimplementovaných procedur a funkcí v jazyce \textsf{R}%
\index{\textsf{R}}.

Ve snaze posílit "edukační" rozměr této práce jsou v relativně rozsáhlém
apendixu uvedeny veškeré zdrojové kódy opatřené volnotextovými komentáři;
kódy postačují pro reprodukci všech uvedených procedur zpracování
textu přirozeného jazyka.

Práce splní svůj smysl, pokud čtenáře zaujme byť i jedna jediná myšlena,
která je v následujícím textu uvedena.


\newpage


%%%%%%%%%%%%%%%%%%%%%%%%%%%%%%%%%%%%%%%%%%%%%%%%%%%%%%%%%%%%%%%%%%%%%%%%%%%%%%%
%%%%%%%%%%%%%%%%%%%%%%%%%%%%%%%%%%%%%%%%%%%%%%%%%%%%%%%%%%%%%%%%%%%%%%%%%%%%%%%
%%%%%%%%%%%%%%%%%%%%%%%%%%%%%%%%%%%%%%%%%%%%%%%%%%%%%%%%%%%%%%%%%%%%%%%%%%%%%%%





