\documentclass[ignorenonframetext,t]{beamer}
\setbeamertemplate{caption}[numbered]
\setbeamertemplate{caption label separator}{: }
\setbeamercolor{caption name}{fg=normal text.fg}
\beamertemplatenavigationsymbolsempty
\usepackage{lmodern}
\usepackage{amssymb,amsmath}
\usepackage{ifxetex,ifluatex}
\usepackage{fixltx2e} % provides \textsubscript
\ifnum 0\ifxetex 1\fi\ifluatex 1\fi=0 % if pdftex
  \usepackage[T1]{fontenc}
  \usepackage[utf8]{inputenc}
\else % if luatex or xelatex
  \ifxetex
    \usepackage{mathspec}
  \else
    \usepackage{fontspec}
  \fi
  \defaultfontfeatures{Ligatures=TeX,Scale=MatchLowercase}
\fi
% use upquote if available, for straight quotes in verbatim environments
\IfFileExists{upquote.sty}{\usepackage{upquote}}{}
% use microtype if available
\IfFileExists{microtype.sty}{%
\usepackage{microtype}
\UseMicrotypeSet[protrusion]{basicmath} % disable protrusion for tt fonts
}{}
\newif\ifbibliography
\hypersetup{
            pdfauthor={Lubomír Štěpánek},
            pdfborder={0 0 0},
            breaklinks=true}

% Prevent slide breaks in the middle of a paragraph:
\widowpenalties 1 10000
\raggedbottom

\AtBeginPart{
  \let\insertpartnumber\relax
  \let\partname\relax
  \frame{\partpage}
}
\AtBeginSection{
  \ifbibliography
  \else
    \let\insertsectionnumber\relax
    \let\sectionname\relax
    \frame{\sectionpage}
  \fi
}
\AtBeginSubsection{
  \let\insertsubsectionnumber\relax
  \let\subsectionname\relax
  \frame{\subsectionpage}
}

\setlength{\parindent}{0pt}
\setlength{\parskip}{6pt plus 2pt minus 1pt}
\setlength{\emergencystretch}{3em}  % prevent overfull lines
\providecommand{\tightlist}{%
  \setlength{\itemsep}{0pt}\setlength{\parskip}{0pt}}
\setcounter{secnumdepth}{0}
\usetheme{Madrid}

% článková matematická notace
\usefonttheme[onlymath]{serif}

% loaduju některé balíčky
\usepackage{siunitx}
\usepackage{amsmath}
\usepackage{physics}
\usepackage{amsfonts}
\usepackage{array}
\usepackage{multirow}
\usepackage{graphicx}
\usepackage[czech]{babel}
\usepackage{blindtext}
\usepackage{scrextend}

% změna velikosti fontu
\newcommand\reduceFontSize{\fontsize{8}{10}\selectfont}

% bude se hodit
\newcommand\floor[1]{\lfloor#1\rfloor}
\newcommand\ceil[1]{\lceil#1\rceil}
\DeclareMathOperator*{\argmax}{arg\,max}

\newcounter{savedenum}
\newcommand*{\saveenum}{\setcounter{savedenum}{\theenumi}}
\newcommand*{\resume}{\setcounter{enumi}{\thesavedenum}}

\defbeamertemplate*{footline}{my footline}
{
    \ifnum \insertpagenumber=1
		\leavevmode%
		\hbox{%
			\pgfsetfillopacity{1}\begin{beamercolorbox}[wd=.23\paperwidth,ht=2.25ex,dp=1ex,center]{author in head/foot}%
			  \usebeamerfont{author in head/foot}\pgfsetfillopacity{1}
			\end{beamercolorbox}%
			\pgfsetfillopacity{1}\begin{beamercolorbox}[wd=.54\paperwidth,ht=2.25ex,dp=1ex,center]{title in head/foot}%
			  \usebeamerfont{title in head/foot}\pgfsetfillopacity{1}
			\end{beamercolorbox}%
			\pgfsetfillopacity{1}\begin{beamercolorbox}[wd=.23\paperwidth,ht=2.25ex,dp=1ex,right]{date in head/foot}%
			  \insertframenumber{} / \inserttotalframenumber\hspace*{2ex}
			\end{beamercolorbox}
		  }%
		\vskip0pt%
    \else
		\leavevmode%
		\hbox{%
			\pgfsetfillopacity{1}\begin{beamercolorbox}[wd=.23\paperwidth,ht=2.25ex,dp=1ex,center]{author in head/foot}%
			  \usebeamerfont{author in head/foot}\pgfsetfillopacity{1}Lubomír Štěpánek
			\end{beamercolorbox}%
			\pgfsetfillopacity{1}\begin{beamercolorbox}[wd=.54\paperwidth,ht=2.25ex,dp=1ex,center]{title in head/foot}%
			  \usebeamerfont{title in head/foot}\pgfsetfillopacity{1}Aplikace \texttt{the\_next\_word\_prediction}
			\end{beamercolorbox}%
			\pgfsetfillopacity{1}\begin{beamercolorbox}[wd=.16\paperwidth,ht=2.25ex,dp=1ex,center]{date in head/foot}%
			  \usebeamerfont{date in head/foot}\pgfsetfillopacity{1}\insertshortdate{}
			\end{beamercolorbox}%
			\pgfsetfillopacity{1}\begin{beamercolorbox}[wd=.07\paperwidth,ht=2.25ex,dp=1ex,right]{date in head/foot}%  
			  \insertframenumber{} / \inserttotalframenumber\hspace*{2ex}
			\end{beamercolorbox}
		  }%
		\vskip0pt%
    \fi
}

\setbeamertemplate{footline}[my footline]

\title{Aplikace\\
\texttt{the\_next\_word\_prediction}}
\subtitle{4IZ470 Dolování znalostí z webu}
\author{Lubomír Štěpánek}
\institute{Katedra biomedicínské informatiky\\
Fakulta biomedicínského inženýrství\\
České vysoké učení technické v Praze\\
---\\
Centrum podpory multimediálních forem výuky\\
Oddělení výpočetní techniky\\
1. lékařská fakulta\\
Univerzita Karlova v Praze}
\date{24. dubna 2017}

\begin{document}
\frame{\titlepage}

\begin{frame}{Pipeline projektu}

\begin{enumerate}
    \item získání textového korpusu
    \begin{itemize}
      \item včetně jeho obohacení vlastním webscrapovaným textem
    \end{itemize}
    \item processing textových dat korpusu
    \item $n$-gramming nad korpusem pro $n \in \{ 2, 3, 4\}$
    \begin{itemize}
      \item včetně Kneserova-Neyova smoothingu
    \end{itemize}
    \item implementace koncové webové aplikace predikující $i$-té slovo, které nejpravděpodobněji následuje uživatelem zadané $(i - 1)$-členné slovní spojení, kde $i \in \{ 1, 2, 3\}$
\end{enumerate}

\end{frame}

\begin{frame}{Získání textového korpusu}

\begin{itemize}
\tightlist
\item
  použita část známých HC korpusů (\textit{Helsinki corpora})
  různorodých anglických textů
\item
  je dostupná online na

  \begin{center}
  \href{http://www.helsinki.fi/varieng/CoRD/corpora/HelsinkiCorpus/}{\framebox{http://www.helsinki.fi/varieng/CoRD/corpora/HelsinkiCorpus/}}
  \end{center}
\item
  pracovní korpus byl konkrétně sestaven

  \begin{itemize}
  \tightlist
  \item
    ze zpravodajských příspěvků
  \item
    z tweetů
  \item
    z blogerských textů
  \end{itemize}
\item
  dohromady \textgreater{} 3 miliony anglických vět
\item
  v plánu obohacení vlastním webscrapingem částí anglicky psaného webu

  \begin{itemize}
  \tightlist
  \item
    twitteru prostřednictvím balíčku \texttt{twitteR} jazyka \textsf{R}
  \item
    Wikipedie, protože nabízí statické HTML
  \end{itemize}
\end{itemize}

\end{frame}

\begin{frame}{Processing textových dat korpusu}

\begin{itemize}
\tightlist
\item
  odstranění větné interpunkce
\item
  odtranění stop slov

  \begin{itemize}
  \tightlist
  \item
    existují slovníky anglických stop slov
  \end{itemize}
\item
  odstranění vulgárních, nevhodných slov

  \begin{itemize}
  \tightlist
  \item
    rovněž pomocí existujících slovníků
  \end{itemize}
\end{itemize}

\end{frame}

\begin{frame}{\(n\)-gramming nad korpusem}

\begin{itemize}
\tightlist
\item
  jde o vytvoření ``slovníku'' \(n\)-členných slovních spojení pro
  \(n \in \{ 2, 3, 4\}\)
\item
  např. \{i like\}, \{how are you\}, \{what about your own\} apod.
\item
  smyslem \(n\)-grammingu je nakonec predikce \(i\)-tého slova, které
  nejpravděpodobněji následuje uživatelem zadané \((i - 1)\)-členné
  slovní spojení, kde \(i \in \{ 1, 2, 3\}\)
\item
  v plánu Kneserovo-Neyovo vyhlazování, jeho principem je provážení
  pravděpodobností \(n\)-gramů pro nízká a vysoká \(n\); momentálně
  implementován MAP
  (\underline{M}aximum-\underline{A}posteriori-\underline{P}robability)
  odhad, tedy slovo \(w_{i}^{*}\) následující frázi
  \(w_{i-1} \ldots w_{1}\) takové, že
\end{itemize}

\[w_{i}^{*} = \argmax\limits_{\forall w_{i}} \{ p( w_{i}^{*} w_{i-1} \ldots w_{1} \mid w_{i-1} \ldots w_{1}) \}\]

\end{frame}

\begin{frame}{Koncová webová aplikace}

\begin{itemize}
\tightlist
\item
  implementována v \textsf{R}, uložena na \textsf{R}-serveru 1. lékařské
  fakulty UK
\item
  beta verze dostupná online na

  \begin{center}
  \href{http://shiny.statest.cz:3838/the_next_word_prediction/}{\framebox{http://shiny.statest.cz:3838/the\_next\_word\_prediction/}}
  \end{center}
\end{itemize}

\end{frame}

\begin{frame}{}

\vspace{+3.0cm}\begin{block}{\centering Děkuji za pozornost!}
  \center
  \begin{tabular}{l@{}l@{}l}
    &\href{mailto:lubomir.stepanek@fbmi.cvut.cz}{lubomir.stepanek@fbmi.cvut.cz} \\
    &\href{mailto:lubomir.stepanek@lf1.cuni.cz}{lubomir.stepanek@lf1.cuni.cz}
  \end{tabular}
\end{block}

\end{frame}

\end{document}
